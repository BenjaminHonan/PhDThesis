% Something about the standard model and ewk gauge bosons. Something something quartic electroweak interactions 
% At scat-tering energies far above the EW scale, the longitudinal modes of the weak bosons are manifestations of the Nambu-Goldstone bosons originating from the spontaneous breaking of EW sym- metry. Probing their interactions therefore helps unveil the dy-namics behind the Higgs mechanism -- https://arxiv.org/pdf/2106.01393.pdf
% is characterized by two high-energetic jets in the forward regions of the detector and reduced jet activity in the central region. The higher center-of-mass energy during the current and subsequent runs strongly boosts the sensitivity in these processes and allows to test the predictions of the Standard Model to a high precisio -- https://arxiv.org/pdf/1610.08420.pdf
%  If the Standard Model (SM) is a partial descrip-tion of Nature and its completion happens at energies which are experimentally unreachable, the cross section of VBS processes could increase substantially be- tween the Higgs boson mass and the scale at which new physics mechanisms intervene to restore the unitarity of the process -- https://arxiv.org/pdf/1801.04203.pdf
The Standard Model (SM) of particle physics is a quantum field theory which describes the interactions of six quarks and six leptons along with their anti-particles in addition to four vector bosons, and a scalar boson. Even though many predictions of the SM have stood up to experimental scrutiny \cite{Intro:expveri}, the theory has a number of shortcomings which indicate it is not a complete description of nature. To name a few of these, the SM does not explain the origin of neutrino masses \cite{Intro:neutrinomasses}, the reason why there are three generations of fermions \cite{Intro:generations}, and the origin of electroweak symmetry breaking \cite{Intro:foot1994explicit,Intro:hierarchy}. It is possible that the resolution to some of these problems lie in the idea that the SM is a low-energy effective theory, which is completed at some higher energy scale \cite{Intro:VBS1,Intro:hierarchy}. It is through precision measurements of SM interactions that this hypothesis can be tested. The study of Vector Boson Scattering (VBS) in particular is of interest due to the presence of quartic EW boson interactions which are potentially sensitive to beyond-the-standard-model (BSM) contributions. Additionally, the scattering cross-sections of VBS processes diverge at high centre-of-mass energies without the presence of delicate cancellations in the form of Higgs boson interactions \cite{Intro:higgs1,Intro:higgs2}. Measurements of VBS processes therefore also provide an insight into the nature of the Higgs mechanism. 

Experimental measurements of VBS processes are extremely challenging because these represent some of the rarest SM processes measurable with the 140\invfb ATLAS Run-2 dataset \cite{Theory:AtlasSummary}. This means that the accuracy of these measurements are usually limited by the statistical precision of the dataset. Furthermore, these measurements are often accompanied by a large irreducible background which dominates over the signal process. Measuring the signal therefore requires precise modelling of this background, in addition to the careful derivation of data-driven background constraints.

The work presented in this thesis focusses on the differential cross-section measurements of $W(\rightarrow\ell\nu)\gamma$ in association with two jets in the electroweak production mode (no strong coupling vertices at leading order).  Separating the signal from the dominant strong irreducible background is facilitated by performing the measurement in a phase-space optimised for sensitivity to VBS processes. Such a phase-space is constructed using knowledge of the characteristic event topologies of VBS processes, which is summarised by the presence of two jets with large longitudinal momenta and a large separation in rapidity. Furthermore, the lack of colour flow between the partonic initial states results in an absence of hadronic activity in the rapidity interval between these two jets. 

The shape and normalisation of the signal and the strong background predictions are constrained via a binned log-likelihood fit of the data to simulation. The construction of three background-enriched control regions, and careful consideration of the floating parameters in the likelihood help to constrain the background. Differential cross-sections in six observables are obtained after applying corrections for detector effects to the extracted signal yields. 

In addition to the measurements of differential cross-sections, this thesis presents work on the calibration of large-radius jets. The hadronic decays of high transverse momentum heavy particles such as top quarks and W or Z bosons result in a highly collimated decay product, where the angular separation of the outgoing partons is often not large enough to enable the resolution of the individual jets. In these situations, the hadronic decay product can be reconstructed as one large-radius jet. Up until recently, the reconstruction of large-radius jets has been performed using inputs constructed solely from topologically grouped clusters of calorimeter cells \cite{Atlas:largercali,Schramm:2017frb}. Recent advancements have lead to jet definitions derived using inputs from combined track and calorimeter information, which results in jets with improved pileup stability, energy/mass resolution, and substructure performance \cite{Atlas:PFlow,Atlas:TCC,Atlas:UFO}. This thesis outlines the calibration of large-radius jets reconstructed using one of these improved inputs known as ``unified flow objects'' (UFO) \cite{Atlas:UFO}.  %the derivation of residual corrections to the jet energy scale calibration of large-radius jets. Specifically, 

The jet calibration work in this thesis pertains to the derivation of residual corrections to the jet energy scale (JES), known as the in-situ calibration. This correction is applied at end of a chain of calibration steps, where the initial steps are primarily responsible for bringing the energy scale of reconstructed jets to that of simulated \textit{truth} particles. The in-situ calibration uses data-to-simulation ratios to primarily correct for detector effects not captured by simulation when applying the JES to data \cite{Atlas:largercali}. The in-situ calibration itself is derived from the statistical combination of a number of different methods which all exploit the transverse momentum balance of a well measured reference object against a probe jet to define a response which should be close to unity. The calibration presented in this thesis is derived using events where the reference system is comprised of a reconstructed \zee or \zmm decay.

The chapters in this thesis are outlined as follows. Chapter \ref{sec:theory} begins with an introduction to the SM, and culminates in a description of the SM Lagrangian. Subsequently, a section on the formation of hadronic final states begins with an introduction to matrix-elements, Feynman diagrams, and cross-sections, and ends with a description of the techniques employed by event generators to model the non-perturbative physics. The final section outlines the defining theoretical and experimental features of VBS, in addition to a summary of recent VBS measurements made at ATLAS. Chapter \ref{sec:atlasdetector} introduces the LHC and the ATLAS experiment, and contains a detailed description of the subsystems comprising the ATLAS detector. Additionally, this chapter describes the ATLAS trigger and data acquisition systems along with the reconstruction algorithms which go into defining the objects used in ATLAS analyses. Chapter \ref{sec:insitu} describes the author's work on the in-situ calibration of large-radius jets, and Chapter \ref{sec:vbswy} outlines the authors work on the differential cross-section measurements of the electroweak production of $W(\rightarrow\ell\nu)\gamma$ in association with two jets. Finally, Chapter \ref{sec:conclusion} concludes this thesis.

%include quartic electroweak boson interactions  % In addition to providing such an effective field theory framework,  %Precision measurements of rare standard model electroweak interactions such as Vector Boson Scattering (VBS) provide a valuable insight in this context, %particularly as the scattering amplitudes of these processes are likely to be  %This highlights the importance of performing precision measurements of the SM 