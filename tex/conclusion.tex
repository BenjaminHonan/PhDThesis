The first differential cross-section measurements of the electroweak production of $W(\rightarrow\ell\nu)\gamma$ in association with two jets at ATLAS are derived in a VBS enriched phase-space as a function of six different observables sensitive to quartic electroweak gauge couplings. %wo angular observables, \jjdphi and \lepgamdphi, are also expected to be sensitive to CP-violating gauge couplings contributing to the $WW\gamma\gamma$ and $WW\gamma Z$ vertices. 

The differential cross-section measurements are performed by extracting the \ewwy signal and constraining the dominant \qcdwy background simultaneously in a binned log-likelihood fit. Constraints to the background are determined using three signal deficient control regions defined by inverting the \lyxi and \ngapjets cuts defining the signal region. The main background constraint is primarily derived in the high \ngapjets control region, and a residual correction is derived primarily in the high \lyxi control region. The optimum shape of the residual correction factor is determined to be a constant based on arguments related to the stability of the likelihood fit. The final constraint to the \qcdwy background in the signal region ($c\cdot\bl$) ranges from 1.07 to 1.39, and the value of $c$ ranges from 1.06 to 1.22, where these values differ depending on the bin and observable. After extracting the signal, detector unfolding corrections are applied to derive the final differential cross-section measurements.  

A cross-check of the nominal signal extraction method is performed by repeating the signal extraction for a different configuration of the parameters in the likelihood. In this alternative configuration, the primary constraint to the strong background is derived in the high \lyxi region, and the residual correction factor is derived in the high \ngapjets region. The results from this cross-check agree with the expectations that the nominal extracted yields should be similar. In the end, this configuration is not used to derive the final results because the lower yields in the high \lyxi region results in worse statistical resolution of the systematic uncertainties. An additional cross-check of only using one control region to derive the backgrounds constraints confirms that the addition of the residual constraint drastically reduces the modelling uncertainties on the measurement.

Because of the small fiducial cross-section of the signal process, the measurement uncertainties are dominated by the statistical precision of the data. The total data statistical uncertainty on the differential cross-section ranges from 16-50\% depending on the bin of the observable. The systematic uncertainties are dominated by the modelling uncertainties on the \qcdwy background and the statistical precision of the MC simulations, where these range from 15-33\% depending of the bin of the observable. %The measurement uncertainties are dominated by the statistical precision of the data

The differential cross-section measurements are compared to predictions from simulations of \ewwy production using the \SHERPA2.2.12 and \MADGRAPH5 event generators at LO accuracy. The results show better agreement with \MADGRAPH5 at low \mjj and \ptjj, but better agreement with \SHERPA2.2.12 at high \mjj and \ptjj. For the angular observables, the results are seen to agree better with the \SHERPA2.2.12 predictions. The measured \mly and \leppt differential cross-sections display better agreement with the \SHERPA2.2.12 predictions. 

The differential cross-sections shown in this thesis are included in the publication \cite{VBSWy:VBSWy}, and contribute to the derivation of constraints on aQGCs derived through an EFT parametrisation. These results include the first LHC constraints of the $f_{T3}$ and $f_{T4}$ operator couplings, and the measurements of the angular observables \dphisigned and \lepgamdphi may help future EFT efforts in deriving constraints to CP violating gauge couplings.

In addition to the differential cross-section measurements, the \zjets in-situ calibration is performed for large-R UFO jets. This calibration is derived as part of a series of calibrations, which differ in the reference system used to balance the transverse momentum of the probe jet. The \zjets calibration has the superior precision over the other in-situ calibrations in kinematic range of around $200\GeV<\pt^{\text{jet}}<500\GeV$. The total uncertainty on the in-situ calibration factor is around $0.5-1.5\%$ depending on the $Z$ decay channel and the transverse momentum of the large-R jet.

The calibrations derived in this thesis are of direct importance to any future ATLAS physics analysis working on large-radius jets using Run-2 or Run-3 data, and the differential cross-section measurements are likely to be used in future global EFT fit efforts to constrain our knowledge of anomalous quadratic gauge interactions. Furthermore, measurements sensitive to vector boson scattering are of growing importance as future experiments begin to probe larger centre-of-mass energies. The state-of-the-art measurements documented in this thesis therefore form part of an increasingly important body of research surrounding some of the rarest and kinematically extreme electroweak processes in the standard model.
